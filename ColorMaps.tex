% -*- latex -*-

\documentclass[twocolumn]{article}

\title{Divergent Color Maps for Scientific Visualization}

\author{Kenneth~Moreland}

\begin{document}

\maketitle

\begin{abstract}
  One of the most fundamental features of scientific visualization is the
  process of mapping scalar values to colors.  This processes allows us to
  view scalar fields by coloring surfaces and volumes.  Dispite the
  importance of this mapping operation, the majority of scientific
  visualization tools and research are still using a color map that is
  famous for its ineffectiveness: the rainbow color map.  The rainbow color
  map, which na\"{i}vely sweeps through the most saturated colors a display
  can reproduce in the order of the colors in a rainbow, is well known for
  its abilities to obscure data, introduce artifacts, and confuse users.

  Although many other color maps have been proposed and used, none have
  been adopted by the visualization community as a good default in
  scientific visualization.  In this paper we explore the use of divergent
  color maps (sometimes also called bipolar color maps) for use in
  scientific visualization.  We conclude with a divergent color map that
  generally performs well in scientific visualization applications.  This
  color map is a clear replacement for the rainbow color map and can
  hopefully, once and for all, kill the use of the rainbow color map for
  serious scientific visualization applications.
\end{abstract}

\section{Introduction}
\label{sec:Introduction}

At its core, visualization is the process of providing a visual
representation of data.  One of the most fundamental and important aspects
of this process is the mapping of numbers to colors.

\end{document}
