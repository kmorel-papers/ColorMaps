% -*- latex -*-

\documentclass[twocolumn]{article}

\usepackage{amsfonts}
\usepackage{amssymb}
\usepackage{amsmath}
\usepackage{graphicx}
\usepackage{varioref}
\usepackage{fancyvrb}
\usepackage[pdfstartview=FitH]{hyperref}

\title{Divergent Color Maps for Scientific Visualization}

\author{Kenneth~Moreland}

% Commands I use for citing.
\newcommand{\lcite}[1]{~\cite{#1}}
\newcommand{\scite}[1]{~\cite{#1}}

% Avoid putting figures on their own page.
\renewcommand{\textfraction}{0.05}
\renewcommand{\topfraction}{0.95}

% Make sure this is big enough so that only big figures end up on their own
% page but small enough so that if a figure does have to be on its own
% page, it won't push everything to the bottom because it's not big enough
% to have its own page.
\renewcommand{\floatpagefraction}{.75}

\begin{document}

\maketitle

\begin{abstract}
  One of the most fundamental features of scientific visualization is the
  process of mapping scalar values to colors.  This processes allows us to
  view scalar fields by coloring surfaces and volumes.  Dispite the
  importance of this mapping operation, the majority of scientific
  visualization tools and research are still using a color map that is
  famous for its ineffectiveness: the rainbow color map.  The rainbow color
  map, which na\"{i}vely sweeps through the most saturated colors a display
  can reproduce in the order of the colors in a rainbow, is well known for
  its abilities to obscure data, introduce artifacts, and confuse users.

  Although many other color maps have been proposed and used, none have
  been adopted by the visualization community as a good default in
  scientific visualization.  In this paper we explore the use of divergent
  color maps (sometimes also called bipolar color maps) for use in
  scientific visualization.  We conclude with a divergent color map that
  generally performs well in scientific visualization applications.  This
  color map is a clear replacement for the rainbow color map and can
  hopefully, once and for all, kill the use of the rainbow color map for
  serious scientific visualization applications.
\end{abstract}

\section{Introduction}
\label{sec:Introduction}

At its core, visualization is the process of providing a visual
representation of data.  One of the most fundamental and important aspects
of this process is the mapping of numbers to colors.  This mapping allows
us to pseudocolor an image or object based on varying numerical data.
Obviously, the choice of color map is important to allow the viewer to
easily perform the reverse mapping back to scalar values.

\begin{figure}
  \centering
  \fbox{The rainbow color map.}
  \caption{The rainbow color map.  Know thy enemy.}
  \label{fig:RainbowColorMap}
\end{figure}

By far the most common color map used in scientific visualization is the
rainbow color map, shown in Figure~\ref{fig:RainbowColorMap}.  In a recent
review on the use of color map, Borland and Taylor\scite{Borland07} found
that the rainbow color map was used as the default in 8 out of the 9
toolkits they examined.  Borland and Taylor also found that in IEEE
Visualization papers from 2001 to 2005 the rainbow color maps was used 51
percent of the time.

\section{Acknowledgements}


This work was done at Sandia National Laboratories.  Sandia is a
multiprogram laboratory operated by Sandia Corporation, a Lockheed Martin
Company, for the United States Department of Energy'�s National Nuclear
Security Administration under contract DE-AC04-94AL85000.

\bibliographystyle{plain}
\bibliography{ColorMaps}

\end{document}
